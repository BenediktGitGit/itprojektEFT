\documentclass{article}

\usepackage[T1]{fontenc} 
\usepackage[utf8]{inputenc}
\usepackage{ngerman}
\usepackage[a4paper,lmargin={2.5cm},rmargin={2.5cm},tmargin={2cm},bmargin={2.5cm}]{geometry}
                   
\begin{document}
\title{Einrichtung - Infrastruktur für Projektarbeit}
\author{Benedikt Hofrichter}
\maketitle

\section{Einrichten einer Teamkommunikationslösung}

\emph{EFT-11}

\section{Einrichten eines Atlassian-JIRA-Projekts}

\section{Integration eines Kanban-Boards in Atlassian-JIRA}
Die JIRA-Version ist mit der der Erweiterung \emph{Agile} ausgestattet. 
Diese ermöglicht das aufsetzen von Scrum- oder Kanban-Boards. Die 

\section{Einrichten eines Github Repository}
Das Filesharing, das zum einen die Bereitstellung von Tools und Quellcodes, vorwiegend Bogofilter betreffend, den Projektteilnehmern zur Verfügung stellt. Die zweite Hauptaufgabe, die kollaborative schriftliche Ausarbeitung wird von diesem Repository unterstützt. Durch die Versionierungsfunktion von Github kann versehentliches Löschen verhindert werden, sowie ist die Entwicklungshistorie nachvollziehbar. 

\section{Integration von Github in Slack}
Slack bietet die Möglichkeit diverese Tools und Technologien zu integrieren. Wir haben uns entschieden einen github-Channel in Slack hinzuzufügen, der nur dem Zweck der Benachrichtung dient, sobald eine neue Version in das Repository gepusht wurde. Dies erhöht die Effizienz der Projektarbeit.

\section{Einrichtung einer Latex-IDE (Texmaker)}
Es gibt viele LaTeX-Editoren und IDEs. Im Folgenden wird beschrieben was bei der Einrichtung des \footnote{\label{foot:2}http://www.xm1math.net/texmaker/}Texmaker-IDE zu beachten und vorzugehen ist.


\section{Bestpratice - Algorithmus}


\begin{thebibliography}{}

\end{thebibliography}

\end{document}
